\documentclass[12pt]{article}
% Preamble
\usepackage[left=2cm,top=1cm,right=2cm,nohead]{geometry}
\usepackage{amssymb}

% Header
\title{CO487 Assignment 4}
\author{Daniel Burstyn (20206120)}
\date{April 7, 2009}

% Body
\begin{document}
\maketitle
\newcounter{count}
\newcounter{subcount}[count]
\newcounter{subsubcount}[subcount]
\begin{list}{\arabic{count}.}{\usecounter{count}}

% Question 1:
\item \textbf{RSA Signatures}
\begin{list}{\alph{subcount})}{\usecounter{subcount}}

% 1. a)
\item It is immediately obvious that this scheme is existentially forgeable.
The message $m=1$ has the signature $s=1$.

% 1. b)
\item It is very easy to retrieve the private key $d$ with a chosen-message
attack, if you choose the message $m=2$.  After obtaining the signature, the
attacker has to compute successive powers of 2 (mod $n$) until the signature is
reached.  This is very efficient to do since computing the next power of 2
involves only a bit-shift operation, and possibly a modulo operation.  The
number of iterations it took to reach the signature is the private key $d$, and
we have completely broken the scheme.

\end{list}

% Question 2:
\item \textbf{Chinese Remainder Theorem}\\
After step (i) we will have 4 integers: $s,t,m,n$ such that $sm+tn=1$.  Now
observe the following:\\
\begin{eqnarray*}
sm+tn &=& 1\\
sm+tn &\equiv& 1 \bmod m\\
tn &\equiv& 1 \bmod m\\
atn &\equiv& a \bmod m\\
atn+km &\equiv& a \bmod m\\
atn+bsm &\equiv& a \bmod m
\end{eqnarray*}
By the same logic, we can also show that $atn+bsm \equiv b \bmod n$.  So if $x =
atn+bsm$ then $x \equiv a \bmod m$ and $x \equiv b \bmod n$ therefore $x \equiv
atn+bsm \bmod mn$.  Thus, I have shown that the algorithm is correct.

% Question 3:
\item \textbf{Error attack on the RSA signature scheme}
\begin{list}{\alph{subcount})}{\usecounter{subcount}}
% 3. a)
\item In order to prove the correctness, we can show that $m^d \equiv s_p \bmod
p$, and that $m^d \equiv s_q \bmod q$.  With this we can say that $s = m^d$ is
the unique solution mod $pq$.  Thus if we come across an $s$ via the algorithm
discussed, then by the uniqueness, it must be $m^d \bmod n$ which of course is
the RSA signature.  So, below we show that $m^d \equiv s_p \bmod p$:\\

First, let us represent $d$ as $q(p-1)+r$ where $q,r$ are the quotient and
remainder when $d$ is divided by $p-1$, respectively.  Now observe that $d_p =
r$.
\begin{eqnarray*}
m^d &\equiv& m^d \bmod p\\
&\equiv&m^{q(p-1)+r} \bmod p\\
&\equiv&m^{q(p-1)}m^r \bmod p\\
&\equiv&m^r \bmod p \textrm{  (Recall by FLT } m^{p-1} \equiv 1)\\
m^d &\equiv& m^{d_p} \bmod p\\
m^d &\equiv& s_p \bmod p
\end{eqnarray*}
By the same logic, we can also show that $m^d \equiv s_q \bmod q$, and thus by
my arguments above I have proven the correctness of the algorithm.

% 3. b)
\item First recall that repeated square multiplcation has a worst-case running
time of $O(k^3)$ where $k$ is the bitlength of the modulus.  Now, since $n =
pq$, if $i,j,k$ are the bitlengths of $p,q,n$ respectively, then $k \approx i +
j$.  So computing $m^d$ directly takes $O(k^3)$.  The proposed algorithm
involves computing $s_p, s_q$ and also finding $s$.  Computing $s_p$ and $s_q$
will take $O(i^3) + O(j^3)$ and finding $s$ can be done via the algorithm in 2.
through EEA and one modulo operation---a total of $\approx O(i^2)$ if we assume
$p > q$ WLOG.  Since we already have $O(i^3)$ we can discard this term as
negligible.

Now the issue is which is faster: $O(k^3)$ or $O(i^3) + O(j^3)$.  Well since
$k=i+j$ we can see that by the binomial expansion, $O(k^3) > O(i^3) + O(j^3)$.
Although they have the same assymptotic running time, the constant factor
difference makes this proposed method faster.  If $i \approx j$, which is
usually true, then we get that this new algorithm is faster by a factor of about
4.

% 3. c)
\item We know that the smart card is producing $s'$ in the form $s' = s_p'xq +
s_qyp \bmod n$ (see question 2) where $s_p'$ is the erroneous $s_p$.  We also
know that the intended signature on $M$ is $m \equiv s^e \bmod n \equiv (s_pxq +
s_qyp)^e \bmod n$.\\
Observe that in the binomial expansion of this, we get $(s_pxq)^e + (s_qyp)^e +$
a number of terms, of which all have at least one $p$ and one $q$ and are all
thus divisible by $n$.  Therefore we get that $m \equiv (s_pxq)^e + (s_qyp)^e
\bmod n$.\\
So, we have the following:
\begin{eqnarray*}
s' &\equiv& s_p'xq + s_qyp \bmod n\\
s'^e &\equiv& (s_p'xq)^e + (s_qyp)^e \bmod n \textrm{ (For the same reason as
above)}\\
m &\equiv& (s_pxq)^e + (s_qyp)^e \bmod n \textrm{ (As we just got above)}\\
m-s'^e &\equiv& (s_pxq)^e + (s_qyp)^e - (s_p'xq)^e - (s_qyp)^e \bmod n\\
&\equiv& x^eq^e(s_p - s_p')^e \bmod n
\end{eqnarray*}
Now, we can find $m - s'^e$ efficiently since it only requires a hash and one
repeated square multiply.  Now observe that gcd($m-s'^e,n) = q$.  Thus we have
found $q$, and $p$ can subsequently be found very easily as $p = n/q$.\\
There are still of course the issues where $x=p$ or $s_p-s_p'=p$.  These,
however, cannot occur.  First $x=p$ is not possible due to the properties of the
EEA, and secondly since $s_p$ is computed mod $p$, if $s_p-s_p'=p$ then that
would imply that $s_p = s_p'$ which cannot be true since we are given the fact
that $s_p' \neq s_p$.

% 3. d)
\item There are a number of ways to prevent such an attack.  By directly
computing the signature we avoid this particular problem.  By verifying the
signature before transmitting we gain a lot of extra security.  The smart card
itself could be designed to better resist this attack.  The most feasible and
efficient solution, I feel, would be verification before transmitting since it
requires only a small, efficient, software change.
\end{list}

% Question 4
\item \textbf{Discrete Logarithms}\\
In this problem, we have $p=569, g=256, q=71$.  So, for the algorithm, $m=9,
g^{-m} = 372 \bmod p$.\\
For the first step, we compute the table:
\begin{center}
\begin{tabular}{c|c}
$j$ & $g^j \bmod p$\\
\hline
0 & 1\\
2 & 101\\
3 & 251\\
1 & 256\\
5 & 315\\
6 & 411\\
7 & 520\\
4 & 528\\
8 & 543\\
\end{tabular}
\end{center}
Now, we compute $h(g^{-m})^{i}$ for successive $i$s until we come across a value
in our table.  For $i=3$ we find the value 528, and therefore $j=4$.  $a = mi +
j = 9*3 + 4 = 31$ and we have found $a = \log _g 327 = 31$.

% Question 5
\item \textbf{Poor random number generator in DSA}\\
Let $k = k_0^b$ be the $k$ used for the first message, and let $r = (g^k \bmod
p) \bmod q$ be the $r$ used for the first message. Furthermore, let $k_1, k_2,
k_3, r_1, r_2, r_3$ represent the $k$s and $r$s used in the three messages.
Finally, let $s_1, s_2, s_3$ be the signatures on $M, M', M''$, which have SHA-1
hashes of $m, m',$ and $m''$.  Now observe that by the definition of this
scheme:
\begin{center}
\begin{tabular}{ll}
$k_1 = k$ & $r_1 = r$\\
$k_2 = kk_0$ & $r_2 = r^{k_0}$\\
$k_3 = kk_0^2$ & $r_3 = r^{2k_0}$
\end{tabular}
\end{center}
We know from DSA that $s_1 = k_1^{-1}(m + ar_1)$, which implies that $k_1 =
s_1^{-1}(m + ar_1)$.  Similarily, we find that $k_2 = s_2^{-1}(m' + ar_2)$, and
from above we know that $k_2 = k_1k_0$.  Therefore, we can plug in, and get that
$k_0s_1^{-1}(m+ar_1) = s_2^{-1}(m' + ar_2)$.

If we follow the same logic, we can also find that $k_0s_2^{-1}(m'+ar_2) =
s_3^{-1}(m'' + ar_3)$.  Now we can compute $m'$ and $m''$ since they are
simple hashes, and each of $r_i$ are given to us.  This gives us two equations,
in which the only unkowns are $k_0$ and $a$.  Since we have two equations and
two unknowns, we are able to solve, and thus find $a$.

This can be done efficiently since this process only involves finding
multiplicative inverses (efficient) and solving a system of equations (also
efficient).

% Question 6
\item \textbf{Elliptic curve computations}
\begin{list}{\alph{subcount})}{\usecounter{subcount}}
% 6. a)
\item $E(\mathbb{Z}_{11}) = \{(0,4), (0,7), (3,4), (3,7), (4,0), (8,4), (8,7),
(9,2), (9,9), \infty\}$
% 6. b)
\item $\#E(\mathbb{Z}_{11}) = 10$
% 6. c)
\begin{list}{\roman{subsubcount})}{\usecounter{subsubcount}}
% 6. c) i)
\item $(0,7) + (3,4)$: First, $\lambda = \frac{4-7}{3-0} = -1$.
\begin{eqnarray*}
P + Q &=& (\lambda^2 - x_1 - x_2, -y_1 + \lambda(x_1-x_3))\\
&=& (1 - 0 - 3, -7 -(0 - x_3))\\
&=& (-2, -9)\\
&=& (9,2)
\end{eqnarray*}
% 6. c) ii)
\item $(3,4) + (3,7)$: Immediately notice that $Q = -R$, and so $Q+R=\infty$.
% 6. c) iii)
\item $(3,7) + (3,7)$: First, $\lambda = \frac{3(3)^2 + a}{2*7} = \frac{7}{3} =
6$.
\begin{eqnarray*}
R + R &=& (\lambda^2 - 2x_1, -y_1 + \lambda(x_1-x_3))\\
&=& (3 - 2*3, -7 +6(3 - x_3))\\
&=& (-3, -6x_3)\\
&=& (8, 7)
\end{eqnarray*}
% 6. c) iv)
\item $(4,0) + (4,0)$: Immediately notice that $S = -S$ since $y = 0$, so once
again $2S = S + S = \infty$
\end{list}

\end{list}

\end{list}
\end{document}
