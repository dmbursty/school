\documentclass[12pt]{article}
% Preamble
\usepackage[left=2cm,top=1cm,right=2cm,nohead,nofoot]{geometry}

% Header
\title{CO487 Assignment 3}
\author{Daniel Burstyn (20206120)}
\date{Feb 27, 2009}

% Body
\begin{document}
\maketitle
\newcounter{count}
\newcounter{subcount}[count]
\begin{list}{\arabic{count}.}{\usecounter{count}}

% Question 1: MAC schemes from block ciphers
\item \textbf{MAC schemes derrived from block ciphers}
\begin{list}{\roman{subcount})}{\usecounter{subcount}}

% 1. i)
\item This MAC scheme is \textbf{insecure}.\\
First recall that a MAC scheme is insecure if it is vulnerable to a
chosen-message attack.  Now I will show that this scheme is vulnerable to such
an attack.

Observe that to compute the tag of any arbitrary message $m = m_1m_2\dots m_t$
an attacker can do the following:\\
- Obtain the message tag pair $(m_1,MAC_k(m_1))$ via chosen-message attack.\\
- From the scheme, we know that $MAC_k(m_1) = c_t = c_0 \oplus m_1 \oplus
E_k(m_1)$\\
- So, compute $E_k(m_1) = MAC_k(m_1) \oplus m_1$\\
- Now repeat this process to determine $E_k(m_2), \dots, E_k(m_t)$\\
- It is now very easy to compute $MAC_k(m)$ since we know all of $m_i$ and
$E_k(m_i)$\\
Thus this MAC scheme is insecure.

% 1. ii)
\item This MAC scheme is also \textbf{insecure}.\\
An attacker can obtain the MAC tag for a 1 block message $m = m_1$ to receive
$(m, c_0, c_t)$.  However, observe that $c_t = E_k(c_0 \oplus m)$.  Also
observe, that if we pick $m' = c_0$, and $c_0' = m$ then $c_t' = E_k(c_0' \oplus
m') = E_k(m \oplus c_0) = c_t$.  Thus $(c_0, m, c_t)$ is also valid, and we have
proved that the scheme is insecure.

% 1. iii)
\item This MAC scheme is also \textbf{insecure}.\\
Since $H$ is a hash function with 80 bit hashes, then we can use the generic
attack for finding collisions (slide 130) to find a collision in $H$ in $\approx
2^{40}$ steps, which is feasible.  Suppose the messages $m$ and $m'$ are the
collision, such that $H(m) = H(m')$.  We can choose the message $m$ to obtain
$MAC_k(m)$.  Again, observe that $MAC_k(m) = E_k(H(m)) = E_k(H(m')) = MAC_k(m')$
and we have thus computed the tag of a new message, and have therefore shown
that the MAC scheme is insecure.

%1. iv)
\item This MAC scheme is also \textbf{insecure}.\\
First we notice that for one block messages, this scheme is almost identical to
the scheme used in (ii), except that the resulting $c_t$ is hashed with $H$, so
we will reuse the attack from (ii).  From (ii) we know that if $(m, c_0, c_t)$
is valid, then $(c_0, m, c_t)$ is valid.  Since $c_t$ remains the same, then for
this scheme, $h$ will remain the same.  So, if we obtain $(m, c_0, h)$ from a
chosen-message attack, then $(c_0, m, h)$ is also valid, and we have once again
proved the scheme is insecure.
\end{list}

% Question 2: Another MAC scheme derived from a block cipher
\item \textbf{Another MAC scheme derrived from a block cipher}

First, let us define $M_i' = M_i$ with the last block of zeros, and the
last one removed.  That way by using the message $M_i'$, the first step of the
encryption will result in a one block modified message $M_i$.  Also, for
convenience let us define $M_0 = 1000\dots 0$ --- the one block message of 1 one
and 127 zeroes.

Now, let's begin by obtaining the MAC tags of $M_1', M_2', \dots, M_l'$.  Which
from the scheme are equal to $c_1, c_2, \dots, c_l$ respectively.

Next, we will obtain $b$ by obtaining the tag to the message $M_1M_1'$.  We see
that $MAC_{(k,b)}(M_1M_1') = c_1(1 + b)$ mod $p$.  Now, since $p$ is prime, we
can divide by $c_1$ (Recall that modulo division is multiplication by an inverse
which ALWAYS exists as long as the modulus is prime.)  Therefore we can obtain
$b$ after a trivial subtraction by 1.

The last thing we need to find is $c_0 = AES_k(M_0)$.  To do this we obtain the
tag for the message $M_1$ (Not $M_1'$).  The padding step of the scheme will
result in the modified message $M_1M_0$, and the tag itself will be $c_1 + c_0b$
mod $p$.  We can subtract $c_1$, and now divide by $b$ to get $c_0$.

We now have all the components we need to compute the MAC of $M$.  Observe that
$MAC_{(k,b)}(M) = (c_1 + c_2b + c_3b^2 + \dots + c_lb^{l-1} + c_0b^l)$ mod $p$,
and that we know all the variables and can thus compute the MAC.

Last we must consider whether this computation is \textbf{efficient}.  The
computations done are addition, subtraction, multiplication (by inverses), and
exponentiation.  Addition, subtraction and multiplication are trivially
efficient, and we know from class that modular exponentiation is also effienct.
Therefore it has been shown that an attacker can \textbf{effienctly} compute he
MAC tag of $M$.

% Question 3: RSA
\item \textbf{RSA computations}
\begin{list}{\alph{subcount})}{\usecounter{subcount}}
\item We must determine Alice's private key from her public key by factoring
$n$:\\
- With a calculator, we quickly find that $1073 = 29 * 37$.\\
- We then compute $\phi = (p-1)(q-1) = 28 * 36 = 1008$.\\
- Since $ed \equiv 1$ mod $\phi$, we can use the extended euclidian algorithm to
determine $d$.  We can use the EEA to find the multiplicative inverse of $e$ in
the integers mod $\phi$.  Let's call the inverse $i$.  Now $ed \equiv 1$ mod
$\phi \Rightarrow ied \equiv i$ mod $\phi \Rightarrow d \equiv i$ mod $\phi$.
Since $d < \phi$ we can conclude that $d = i$.\\
- We will use the EEA to find $x$ and $y$ in $\phi x + ey = 1$ since $e$ and
$\phi$ are coprime, and in fact $y$ is the multiplicative inverse we seek.\\
\\
\begin{tabular}{|c|c|c|c|}
\hline
$x$ & $y$ & $ex + \phi y$ & quotient\\
\hline
1 & 0 & 1008 & \\
0 & 1 & 715 & 1 \\
1 & -1 & 293 & 2 \\
-2 & 3 & 129 & 2 \\
5 & -7 & 35 & 3 \\
-17 & 24 & 24 & 1 \\
22 & -31 & 11 & 2 \\
-61 & 86 & 2 & 5 \\
327 & -461 & 1 & \\
\hline
\end{tabular}\\
\\
- Using the EEA we find that $y = d = -461 \pmod \phi \equiv 547 \pmod \phi$ and
thus we have obtained Alice's private key.

\item To encrypt $m = 3$, we begin by obtaining $(n,e) = (1073, 715)$ from the
assignment.  We now need to compute $c = m^e$ mod $n$.  We can use the repeated
exponentiation process to do this computation effeciently, although doing it by
hand can be painstaking, so we will do the computation in Maple instead.  We
find that $c = 548$.

\end{list}

\newpage
% Question 4: GCDeasy
\item \textbf{Computing GCD without long division}
\begin{list}{\alph{subcount})}{\usecounter{subcount}}

% 4. a)
\item Compute GCD using GCDEasy: \\
\begin{tabular}{|l|c|c|c|}
\hline
Step & $a$ & $b$ & $e$ \\
\hline
Begin & 308 & 440 & 1\\
While $a,b$ both even... & 154 & 220 & 2\\
... & 77 & 110 & 4\\
While $a \neq 0$ \{ & & & \\
\ \ While $b$ is even... & 77 & 55 & 4\\
\ \ $a \ge b$ so $a = a-b$ & 22 & 55 & 4\\
\ \ goto start of loop & & & \\
\ \ While $a$ is even... & 11 & 55 & 4\\
\ \ $b > a$ so $b = b-a$ & 11 & 44 & 4\\
\ \ goto start of loop & & & \\
\ \ While $b$ is even... & 11 & 11 & 4\\
\ \ $a \ge b$ so $a = a-b$ & 0 & 11 & 4\\
\} & & & \\
\hline
\end{tabular}\\
\\
The result is $e*b = 44$.

% 4. b)
\item The first step of GCDeasy is while $a$ and $b$ are both even, we divide
them by 2.  Since $a$ and $b$ are both integers, they can be prime factored.
The loop effectively is while there is a 2 in the prime factorization, remove a
2.  It is obvious to see that eventually, all the 2s will be gone, since both
numbers are finite, and the loop will end, and we will move on to the next
step.

For the next step, at the beginning of each loop, we reduce $a$ and $b$ so that
they are both odd.  Now observe that the if statement will either reduce $a$ or
reduce $b$, and that $a$ cannot go below 0, and that $b$ cannot go below 1.

We can see that at each iteration, either $a$ or $b$ must decrease, that
neither can increase at any point, and that neither can go below 0.  By this
logic, we can conclude that either $a$ or $b$ will eventually reach their
minimum values.  If $a$ reaches 0, then the loop obviously terminates.  If $b$
reaches 1, then we can easily see that $a$ will continue to decrease until it
reaches 0, and the loop will still terminate.

Therefore this procedure always terminates.

% 4. c)
\item First lets simplify a bit by considering the first loop, and last step.
Note that at the end of the first loop, $e$ is the greatest common divisor or
$a$ and $b$ that is a multiple of 2.  Thus we can see that
gcd$(\frac{a}{e},\frac{b}{e}) * e = $ gcd$(a,b)$.  Now we only need to worry
about the correctness of the inner loop.  For convenience let $a' = \frac{a}{e}$
and $b' = \frac{b}{e}$.

Let us now say that $\hat a$ and $\hat b$ refer to the running values of the
variables in the procedure.  At the top of the loop, we know of course that
gcd$(a',b')$ = gcd$(\hat a,\hat b)$.  This is our loop invariant.
we know that we have already taken out all the 2s from their gcd, so by dividing
by 2 here again, we can still say that our invariant is true.  Now the if
statement either does $\hat a = \hat a - \hat b$ or $\hat b = \hat b - \hat a$
depending on which is larger.  However, from the assignment, we know that
gcd$(a,b) = $ gcd$(a, b-a)$, and that similarily gcd$(a,b) = $ gcd$(a-b,b)$,
again depending on which is larger.  So, once again we can conclude that our
invariant is still true.  Now that we have reached the end of the loop, and the
invariant still holds true, we can conclude that after the loop finishes (which
it does as proved in (b)) that gcd$(a',b')$ = gcd$(\hat a,\hat b)$.  Since $\hat
a = 0$, we can conclude that $gcd(\hat a, \hat b) = \hat b$.

Now that we have shown that inner loop is correct, and that the initial loop,
and the final multiplication by $e$ are correct if and only if the inner loop is
correct, we can conclude that the procedure in its entirety is correct.

% 4. d)
\item First, we observe that each division is a bitshift (one bit operation) and
each subtraction is single bit operation, so the running time of GCDeasy is the
number of divisions plus the number of subtractions.

In the worst case, $a$ is $\{1\}^k = 2^{k}-1$ (and is odd), and to minimize the
number of subtractions, we want $b$ to be one in the inner loop, so the worst
case is if $b$ is $\{1\}\{0\}^{k-1} = 2^{k-1}$.  Note that both numbers have the
same bitlength ($k$) and that it doesn't matter which is $a$ or which is $b$ so
I picked them this way arbitrarily.

In the first step, $a$ is odd, so we immediately move on to the next step.  Here
$b$ is even, so we reduce it to 1 in $k-1$ divisions (since it was $2^{k-1}$).
Notice now that $a = \{1\}^k$ and that $b = 1$.  After the if statement, $a$ is
reduced by one and is now $\{1\}^{k-1}\{0\}$ for a total of $k$ bit operations
so far.

Now, $b$ will remain at 1 for the remainder of the process.  We observe that at
each iteration $a$ is now even, so it is bitshifted by one, and then again the
if statement is reduced by 1 and now $a$ looks like $\{1\}^{k-2}\{0\}$.  This
will repeat until $a = 0$ which means two bit operations times $k-1$ bits.  This
gives a grand total of $3k-2$ bit operations, and thus the running time for
GCDeasy is $O(k)$.

% 4. e)
\item From (d) we found that the running time for GCDeasy is $O(k)$.  We can see
that a $k$-bit number is at most $2^k-1$, and so, if the running time of GCDeasy
is $O(k)$, then the running time is $O(log_2(a))$ where $a$ is the larger of the
two numbers.  We know that log time is polynomial-time, so \textbf{yes}, GCDeasy
is a polynomial-time algorithm.

\end{list}

\end{list}
\end{document}
