\documentclass[titlepage]{article}
\usepackage[left=2.54cm,top=2.54cm,right=2.54cm,nohead]{geometry}
\usepackage{graphicx}
\setlength{\parskip}{2mm}
\title{User Manual for \textsf{soblogitsgood}}
\author{Yubin Kim, Daniel Burstyn, Gobaan Raveendran, Nathaniel Flath}
\begin{document}
\maketitle
\newpage
\tableofcontents
\newpage

\section{Introduction}

\subsection{Outline}
This document will first cover what our product, \textsf{soblogitsgood},
does, and the motivation for creating it.  Common, simple ways to use
it are covered next, along with descriptions of each of our screens.
We then cover the advanced features and how to use our developer API.
Finally, a glossary of terms is presented.

\subsection{Product Overview and Motivation}
\ \ \ \ \textsf{soblogitsgood} at its core is a large system that crawls the internet and
collects blog information.  Blogs in particular are an interesting data source
because, unlike most web pages, they are typically written by a single person
from their point of view.  If we filter out this particular data we can
investigate what the big topics the blogging population (the "blogosphere") is
talking about.

\textsf{soblogitsgood} can be used in two very distinct ways.  The first is
its web interface, which allows users to query the database looking for 
\textbf{opinions} on a topic.  A typical Google search will typically return a
wide range of informative results.  However, user will frequently want to hear
about a product or service from another actual person.  \textsf{soblogitsgood}
can be used to find user stories and not just feature lists.

The second use is more advanced and aims to provide access to the service
through an API. This API is available to developers who want programmatic
access to our service. Also available is a data access API for those who want
to perform their own analysis on our corpus of blogs.  
This saves them the trouble of having to collect and clean the blog data.


\section{Search Basics}
Search is simple: just type whatever comes to mind in the search box, hit
Enter or click on the Search button, and \textsf{soblogitsgood} will search the
blogosphere for blogs that are relevant to your query.  Unfortunately, due to
the nature of natural languages, it may be hard to get exactly the results you
were looking for without refinement. Thus, the following tips can help you
refine your technique to make the most of your searches.

\subsection{Some Basic Tips}
\begin{itemize}
\item \textbf{Select a category.} Certain terms in natural languages become
almost impossible to disambiguate, for example logs about opera's can either
be about the theatrical performances, or the browser, or many other words.
Furthermore, results can be obtained faster and more filter functions can be
used when a category is selected, due to the nature of the search algorithms
used by \textsf{soblogitsgood}.
\item \textbf{Search is always case insensitive.} Searching for “new york times” is the
same as searching for “New York Times”.
\end{itemize}

\subsubsection{How to read search results}
\textsf{soblogitsgood}'s goal is to provide you with results that are clear and easy to
read. The diagram below points out four features that are important to
understanding the search results page:
\begin{center}
\includegraphics{results.png}
\end{center}

\begin{enumerate}
\item The Title: The first line of any search result is the title of the
webpage.
\item The Snippet: A description of or an excerpt from the webpage.
\item The Sentiment: The basic stance the article has about the topic.
\item Keyword: The category this blog falls under.
\item Bins: Summary of categories for each result.
\end{enumerate}

All these features are important in determining whether the page is what you
need. The title is what the author of the page designated as the best short
description of the page.

The snippet is \textsf{soblogitsgood}'s algorithmic attempt to extract just the part of the
page most relevant to your query. The sentiment plus keywords help you figure
out if the article is what you want. Keywords help you verify that the link is
the correct topic, and the bins will allow you to refine the results if you
find that there are to many options, and many are irrelevant.

\subsection{Break It Down}
If a finer level of detail is wanted, press `Break It Down'. This mode causes
all the data to be organized into good and bad results. This allows you to
gather a quick overview of the system, and get an overall idea of how positive
and negative the results are.

\subsubsection{How to read the breakdown}
The following screen helps us determine how the breakdown works in order to
get a quick overview of the system.
\begin{center}
\includegraphics{brokendown.png}
\end{center}

The breakdown consists of two components:
\begin{enumerate}
\item Visualization
\item Result separation.
\end{enumerate}

The visualization component consists of a simple pie graph that will allow you
to quickly determine if the overall sentiment is for or against the given
topic.  The separation component allows us to isolate one set of opinionated
results, and by clicking a sentiment we can display only the results that
match the sentiment.


\subsection{Help Mode}
Users can get help with our product by clicking a `help' link on any of our
pages.  This link redirects to help on the page they just came from; if
clicked on our main page, it will explain the purpose of our product and how
to use it.  This `main' help page will also be accessible from all other help
pages.  If `help' is clicked on a results page, the opened page will also
explain how to interpret the results presented.

The main help page has basic information about our product and how to
use it.  It additionally links to this document, or an updated user
manual.  The help navigation tree is on the left, allowing the user to
easily access help on any page they desire.

Each analysis and results page has a help page.  Each feature on the
page has an explanation - each visualization, column, etc. has it's
relevant features pointed out.  This page describes to the user
precisely what they are looking at and what they can do in order to
clear any confusion they may have.

\section{Advanced Features}
For the more advanced user, we provide APIs for making sentiment search
queries, and for getting access to our raw data.  We will write and make
available language-dependent APIs for this.  Currently a Java API is available
and a language independent API using HTTP requests will be made available
soon.


\subsection{Sentiment Search API}
The Sentiment Search API has the exact same function as \textsf{soblogitsgood}'s web
interface but can be accessed programmatically.  This is useful for advanced
users that wish to retrieve blog post information and do their own analysis on
top of to existing sentiment analysis.  The Java API for this supports the
following methods:

\begin{verbatim}
BlogResults sentimentSearch(String query);
BlogResults sentimentSearch(String query, SearchFilter filter);
\end{verbatim}

BlogResults support methods to retrieve blog posts, and SearchFilter can be
used to specify restrictions on the results like posting data, or category.

Here is an example of how to use the API:
\begin{verbatim}
SearchFilter filter = new SearchFilter();
filter.setCategory(SentimentSearch.Catergories.TECHNOLOGY);
BlogResults results = SentimentSearch.sentimentSearch("iPhone", filter);
while(results.hasMoreResults()) {
  System.out.println(results.getNextBlogPost().getURL());
}
\end{verbatim}

\subsection{Data Access API}
The Data Access API allows developers to get direct access to \textsf{soblogitsgood}'s
database of blog information.  This API will have access restrictions to avoid
the general public from flooding our system with requests for huge numbers of
results.  The Data Access API is very similar to the sentiment search one, but
has less limitations.  The supported Java API methods are:

\begin{verbatim}
BlogResults getBlogs(SearchFilter filter);
BlogResults getRandomBlog();
BlogResults getRandomBlog(SearchFilter filter);
BlogPost getRandomBlogPost();
BlogPost getRandomBlogPost(SearchFilter filter);
\end{verbatim}

We allow users to request all blog posts that match the specified filter.
This is useful if they want to do an analysis over the large data set.  We
also support getting random blogs or blog posts so that developers can do an
analysis on a smaller sample of our data.

\section{Glossary}
\begin{description}
\item[API]
API stands for Application Programming Interface and is an interface that a
piece of software implements that allows other software to interact with it.
\item[Bin]
Bins are categorizations that [The Product] uses to classify articles. These
categories mirror the structure of data within our servers, and thus searches
within a bin are much faster and more accurate.
\item[Blog]
A blog is a "web log" that consists of the authors thoughts, sentiments or
general ideas. They can express anything from the day to day activities to a
highly technical tutorial about a given topic.
\item[Blogosphere]
The blogosphere is made up of all blogs and their interconnections. The term
implies that blogs exist together as a connected community (or as a collection
of connected communities) or as a social network in which everyday authors can
publish their opinions.
\item[Corpus]
A corpus is a collection of writing, that may be preprocessed and annotated to
enhance automatic processing, testing, or verification of algorithms.
\item[Crawl]
A crawler explores the web in some kind of automated manner in order to discover
sites that match a certain criteria. In \textsf{soblogitsgood}'s use case, all sites are
explored but only blogs are stored and processed.
\item[Natural Language]
Natural languages are languages that are not formally defined and instead
evolved over time through every day human interaction. Currently, English is
the main natural language that we search using \textsf{soblogitsgood}.
\item[Opinion]
In our case opinion is basically any kind of natural discussion about a given
topic. Although opinions should be based on facts, in general they do not have
to be, and in general an opinionated article should not be affiliated with the
topic is is discussing, to avoid bias.
\item[Sentiment]
Sentiment is the overall emotional content within an article. Some articles
for example will be angry and others may be gleeful. Some sentiments can be 
classified into positive and negative allowing for refinement.
\end{description}
\end{document}

p
