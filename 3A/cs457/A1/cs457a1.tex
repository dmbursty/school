\documentclass[12pt]{article}
% Preamble
\usepackage[left=2cm,top=2cm,right=2cm,nohead,nofoot]{geometry}
\usepackage{enumerate}
\usepackage{setspace}

% Header
\title{CS457 Assignment 1}
\author{Daniel Burstyn (20206120)}
\date{January 30, 2009}

% Body
\begin{document}
\maketitle
\tableofcontents
\doublespace
\newpage
\addtocounter{section}{1} % Start at section 2
\section{Services and Outcomes}
  \subsection{Services}
    \subsubsection{Process update packet locally}
The first service, is one that faces the user directly.  This service is
processing of user's update packets, and making the necessary changes to the
local server.  This conveys the user's new location to all users that share that
local server.  Ideally, this when a user sends an update packet, this is the
only service that needs to be used, but sometimes the system needs to go one
step further.
    \subsubsection{Update location to remote servers}
Occasionally when a user sends an update packet, there are other users in his
vision domain that need to be notified, but are not on the local server.  In
this case, the user's local server needs to send a syn packet to various other
remote servers.  This is the second service of the system.
  \subsection{Outcomes}
    \subsubsection{Success}
Since the failure of servers is outside the scope of this assignment, success
is the only possible outcome of our services.
\section{Performance Metrics}
  \subsection{Scalability}
The first set of performance metrics are those that measure scalability.
Scalability is something that itself is very hard to measure, so we must pick
our metrics carefully.
    \subsubsection{Utilization}
Utilization is defined as the proportion of time that a server is busy.  The
lower the utilization, the more time the server is idle.  We want to minimize
the time that the server is idle (and thus maximize utilization) because it is
not cost effective to purchase servers that sit idle, and for the system to be
scalable, we need to minimize cost.  This can be considered in both a individual
and global sense in terms of when a specific server is busy, and all servers
aggregated.
    \subsubsection{Latency}
Latency refers to the time between the user sends his update packet, and when
his local server begins processing the request.  With many users, the queue on
servers can become quite large, causing latency to increase.  This is a
individual metric because it is specific to each user's local server.
    \subsubsection{Number of syn packets per update}
This is the average number of syn packets that a server has to send when it
receives an update packet.  This is a global metric because it is averaged
across the whole system.
The sending of syn packets to other remote servers is the most expensive task to
perform for a server.  As population grows, so does the density within the VE,
and thus the number of syn packets that need to be sent when the average user
moves also increases.  In order for the system to be scalable, we want to
minimize the number of syn packets that need to be sent.
  \subsection{Consistency}
The other aspect of the system that is important to us is user state
consistency.  There are a number of metrics that help us determine how
consistent the system is on average.
    \subsubsection{Duration of inconsistent data}
This is the amount of time that the system is in an inconsistent state.  Since
this deals with the entire system's state, it is a global metric.  Using
the same notation as the textbook, this time is $\approx $ max$(D_{l1}, D_{l2},
\dots, D_{ln})$  Where $l$ is the local server out of $n$ servers.  Creating and
sending of syn packets is very simple is fast enough to be negligible in this
situation.
This is an important metric that we absolutely want to minimize because it
affects how realistic the VE interactions are.  Delay between users seeing
eachothers actions causes laggy interactions that are bad for the user's
experience.
    \subsubsection{Number of remote servers causing inconsistency}
This is the number of remote servers that require syn packets upon a user
movement so that the system can maintain a consistent state.  Again, since this
is measured across the whole system, it is thus a global metric.  This is an
important metric because it is the largest bottleneck in maintaining consistency
across the system.
\section{System Parameters and Workload Parameters}
  \subsection{System Parameters}
    \subsubsection{Number of servers}
It is fairly obvious that the number of servers available to the system is an
important parameter and has an impact on a number of performance metrics.
    \subsubsection{Server capacity}
Server capacity is the rate at which a server can process user requests.  This
affects how many users a server can handle, and thus how soon new servers must
be purchased.
    \subsubsection{Network latency}
This includes both the time it takes for a user to send an update packet to
their local server ($d_{au}$), and the time it takes to transmit a syn packet
from one server to another ($D_{ab}$).  This is important as it affects how long
the system is in an inconsistent state.
    \subsubsection{Syn service requirement}
This is the time it takes for a remote server to service a syn packet and update
it's state.
    \subsubsection{Update service requirement}
This is the time it takes for a user's local server to process an update
request, including creating and sending off necessary syn packets.
  \subsection{Workload Parameters}
    \subsubsection{Number of users}
The number of users is perhaps the most important parameter because it specifies
the scale of the system.
    \subsubsection{Service requirement of users}
This refers to the average service requirement of a user in terms of how often
they send update packets, and how often those update packets require syn packets
to be sent to remote servers.  For example if users tended to all stay close
together, a higher proportion of update packets would require syn packets as
well.
\end{document}
