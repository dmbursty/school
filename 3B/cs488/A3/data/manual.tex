\documentclass[12pt]{article}
% Preamble
\usepackage[left=2cm,top=2cm,right=2cm,nohead,nofoot]{geometry}
\pagestyle{empty}
 
% Header
\title{CS488 Assignment 3}
\author{Daniel Burstyn (dmbursty) (20206120)}
\date{Oct 28, 2009}
 
% Body
\begin{document}
\begin{titlepage}
\maketitle
\thispagestyle{empty}
\end{titlepage}
\newpage

\begin{center}
\begin{Huge}
Manual
\end{Huge}
\end{center}

\section{Head Joint}
In order to prevent the head from being inside the body, I have joint nodes
store their rotation amounts (which they have to do anyways to enforce min/max),
and upon walk\_gl, I use glRotated to actually do those rotations instead of
modifying m\_trans directly.

\section{Picking}
When a primitive is picked, the scene is traversed to find the first joint node
parent of that primitive.  Then, all of that joint's children are picked.  This
is so that if a user clicks on the puppet's nose, the head and eyes will also be
picked.

A picked primitive's colour will change to green.  To avoid confusion, the
puppet I submitted has no green primitives.

\section{Undo Stack}
When a user presses either mouse2 or mouse3, an undo "scope" is started.  That
scope is terminated at the point where the user releases \textbf{both} mouse2 and
mouse3.  If the user does the following actions: press mouse 2, drag, press
mouse 3, drag, release mouse2, drag, release mouse3, then only one item will be
added to the undo stack.

If the user clicks and does not move the mouse, then the undo stack is left
unchanged.

\end{document}
