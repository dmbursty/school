\documentclass[12pt]{article}
% Preamble
\usepackage[left=2cm,top=1cm,right=2cm,nohead,nofoot]{geometry}

% Header
\title{CO487 Assignment 1}
\author{Daniel Burstyn (20206120)}
\date{January 23, 2009}

% Body
\begin{document}
\maketitle
\newcounter{Lcount}
\begin{list}{\arabic{Lcount}.}{\usecounter{Lcount}}

% Question 1: Viginere Cipher
\item \textbf{Vigin\`{e}re Cipher}\\
The secret key used to encrypt Page20 was \textbf{HAZELNUT}.

The first approach I used to try and decrypt the text was to look at various
frequency distributions of letters based on the possible lengths of the secret
key.  However, due to the small amount of characters, and the complexity of
frequency analysis, I chose to use a different approach.

I investigated the reapeated three letter blocks to see if I could determine the
key length.  Given that the size of the key is relatively small, and that we
have a fair amount of ciphertext, we can conclude that a large proportion of the
repeated blocks are due to the same word being encrypted by the same part of the
key, and not just coincidences.  Since the distance between the repeated blocks
was always a multiple of 8, the key most likely had length 8.

Knowing the length of the key, it was quite simple to retrieve a list of words
in english, extract all 8-letter words with no repeated letters, and perform an
exhaustive key search.

% Question 2: Hill Cipher
\item \textbf{Hill Cipher}
\newcounter{subLcount}
\begin{list}{(\alph{subLcount})}{\usecounter{subLcount}}
% 2. (a)
\item To decrypt ciphertext c, one only needs to compute
$m = D(c) = (c - b)A^{-1}$ modulo 2.

% 2. (b)
\item In a chosen-plaintext attack, the attacker can start by choosing:\\
$m = [ 0,0,...,0 ]$ which gives $c = b$.\\
Now with $b$, we can determine $A$ using $n$ more plaintexts.  From the
encryption function, we can easily see that if $m$ is all 0s except for a 1 in
the $i$th column, then $c$ is the $i$th row of $A$.  So, we can reconstruct $A$
using all of these $n$ plaintexts.

% 2. (c)
\item First observe that by the properties of exclusive or, if $a \oplus b =
c$ then flipping the $i$th bit of $a$ will also flip the $i$th bit of $c$.\\
Now, consider the decryption routine outlined in (a): $D(c) = (c - b)A^{-1}$ mod
2, which is equivalent to $D(c) = (c \oplus b)A^{-1}$ because of the modulo 2.\\
If we take $c$, and flip it's first bit, then $c \oplus b$ will also have it's
first bit flipped.  Now let's investigate what effect this will have on the
matrix multiplication:\\
Suppose we have $d = [ d_1, d_2, ..., d_n ]$ and
$A = \left[\begin{array}{ccc}
A_{11} & \dots & A_{1n}\\
\vdots & \ddots\\
A_{n1} & & A_{nn}\\
\end{array} \right]$ , and $dA = [ e_1, e_2, \dots, e_n ]$\\
By the properties of matrix multiplication, $e_1 = d_1A_{11} + d_2A_{21} + \dots
+ d_nA_{n_1}$ and in general $e_i = \sum^n_{j=1}d_iA_{ji}$.\\
From this we can see that if we were to flip $d_i$, then the new result would be
$e$ with bits flipped in the positions where the $i$th \textbf{row} of $A$ has
1s.

Thus by those two observations, we can conclude that by flipping a bit in the
ciphertext, will result in flipping a distinct set of bits of the plaintext.
Now, producing the real plaintext becomes quite simple:\\
Given the ciphertext $c = [ c_1, c_2, \dots, c_n ]$ the three chosen ciphertexts
should be:\\
\[ c' = [ \overline{c_1}, c_2, \dots, c_n ] \]
\[ c'' = [ c_1, \overline{c_2}, \dots, c_n ] \]
\[ c''' = [ \overline{c_1}, \overline{c_2}, \dots, c_n ] \]
With $m'$, $m''$, and $m'''$ as the respective plaintexts.  Let $U$ be the set
of bits flipped in $m'$, and $V$ be the set of bits flipped in $m''$.  The bits
flipped in $m'''$ should then be $U \cup V - U \cap V$.  Observe that $m' \oplus
m'''$ will be $m$ except with $V$ bits flipped (since it will reflip
the bits in $U \cap V$.  With $m' \oplus m'''$ and $m''$ both having only $V$
bits flipped, xoring them will yield the original plaintext $m$.

In conclusion, in order to find the original plaintext $m$, the \textbf{three}
chosen ciphertexts should be $c' = [ \overline{c_1}, c_2, \dots, c_n ]$,
$c'' = [ c_1, \overline{c_2}, \dots, c_n ]$, and
$c''' = [ \overline{c_1}, \overline{c_2}, \dots, c_n ]$, yielding $m'$, $m''$,
and $m'''$.  $m$ is simply retrieved by computing $m' \oplus m'' \oplus m'''$.
\end{list}

% Question 3: Weakness in RC4
\item \textbf{Weakness in RC4}
\newcounter{subLcount2}
\begin{list}{(\alph{subLcount2})}{\usecounter{subLcount2}}
% 3. (a)
\item To prove this I will walk step by step through the key generating function
while showing the contents of $S$, and the values of $i$ and $j$.  First let us
assume that $S[1] = x \neq 2$, and for convenience I have merged the last two
steps of the function into one.

\begin{center}
\begin{tabular}{| c | c || c | c | c | c | c | c |}
\hline
Step & Command & $i$ & $j$ & $S[1]$ & $S[2]$ & $S[x]$ & Output \\ \hline
0 & begin & 0 & 0 & $x$ & 0 & ? & \\ \hline
1 & $i \leftarrow (i+1)$ mod 256 & 1 & 0 & $x$ & 0 & ? & \\ \hline
2 & $j \leftarrow (S[i]+j)$ mod 256 & 1 & $x$ & $x$ & 0 & ? & \\ \hline
3 & Swap($S[i], S[j]$) & 1 & $x$ & ? & 0 & $x$ & \\ \hline
4 & Output($S[i] + S[j]$) & 1 & $x$ & ? & 0 & $x$ & $S[? + x]$ \\ \hline
\hline
5 & $i \leftarrow (i+1)$ mod 256 & 2 & $x$ & ? & 0 & $x$ & \\ \hline
6 & $j \leftarrow (S[i]+j)$ mod 256 & 2 & $x$ & ? & 0 & $x$ & \\ \hline
7 & Swap($S[i], S[j]$) & 2 & $x$ & ? & $x$ & 0 & \\ \hline
8 & Output($S[i] + S[j]$) & 2 & $x$ & ? & $x$ & 0 & $S[x] = 0$ \\ \hline
\end{tabular}
\end{center}

Therefore, with the given parameters, the second bit of the keystream will
always be 0.

%3. (b)
\item First, assume that the key scheduling algorithm produces a uniformly
random $S$.  That is, the probability of any $S[i]$ being a certian number is
the same as it being any other number.  I estimated that this is at least
approximately true using a large sample of keys.  Similiarily, assume that
except for the second byte, all other bytes have equal probability of being 0,
and that all other numbers have a uniform distribution.  This was also verified
using a large sample of keys.  Although an estimation, it is a safe assumtion to
make given that we are only attempting to \textbf{approximate} the probability
of the second keystream byte being 0.

From our assumptions, it is fairly simple to calculate the probability of the
second byte of the keystream being 0:\\
$p[$second byte is 0$] = (p[S[2] = 0] * p[S[1] \neq 2]) + (p[S[2] \neq 0] *
p[S[x] = 0])$\\
Where $x = S[2] + S[S[1]+S[2]]$ (this is from stepping through the key
generating algorithm like in (a))

The first part: $p[S[2] = 0] * p[S[1] \neq 2]$ comes from (a).  Breaking it up
we get $p[S[2] = 0] = \frac{1}{256}$ since the key scheduling algorithm is
uniformly random, and $p[S[1] \neq 2] = \frac{255}{256}$ for the same reason.

The second part comes from how 0 would be the output if $S[2]$ was not 0.
Breaking it up we get $p[S[2] \neq 0] = \frac{255}{256}$ again because the key
scheduling algorithm is uniformly random.  $p[S[x] = 0] \approx \frac{1}{256}$
because all other bytes have equal probability of being 0.

Putting that all together, we get $p[$second byte is 0$] \approx
\frac{1}{256} * \frac{255}{256} + \frac{255}{256} * \frac{1}{256} \approx
\frac{1}{128}$\\
This is about twice as probable as any other value, since all other values have
equal ($\approx \frac{1}{256}$) chance of being the second keystream byte.

%3. (c)
\item Since we know that the second byte of the keystream is biased towards 0,
if we gathered the second bytes of all of the ciphertexts and found the byte
that occurred the most often, it would most likely have been encrypted with 0.
Given the probabilities from (b), we can expect to see twice as many of these
bytes than any other, which is ``a good chance" of being right.  Since that byte
would have been encrypted with 0 (via $\oplus$) that byte is the plaintext byte
that we seek.
\end{list}

% Question 4: Feistel ciphers
\item \textbf{Feistel ciphers}
\newcounter{subLcount3}
\begin{list}{(\alph{subLcount3})}{\usecounter{subLcount3}}
% 4. (a)
\item Decryption also takes $h$ rounds:\\
Round 1: $(m_h, m_{h+1}) \rightarrow (m_{h-1}, m_h)$, where $m_{h-1} =
m_{h+1} \oplus f_h(m_h)$.\\
Round 2: $(m_{h-1}, m_h) \rightarrow (m_{h-2}, m_{h-1})$, where $m_{h-2} =
m_h \oplus f_{h-1}(m_{h-1})$.\\
\dots \dots \\
Round $h$: $(m_1, m_2) \rightarrow (m_0, m_1)$, where $m_0 =
m_2 \oplus f_1(m_1)$.

% 4. (b)
\item If we have a plaintext/ciphertext pair $(m, c)$, where $m = (m_0, m_1), c
= (m_2, m_3)$ it is very easy to break this encryption and find the secret key.
Let's look at the result of the encryption process through the feistel ladder:\\
Round 1: $(m_0, m_1) \rightarrow (m_1, m_2)$, where $m_2 = m_0 \oplus f_1(m_1) =
m_0 \oplus m_1 \oplus k_1$\\
Round 2: $(m_1, m_2) \rightarrow c = (m_2, m_3)$, where $m_3 = m_1 \oplus
f_2(m_2) = m_1 \oplus m_2 \oplus k_2 = m_1 \oplus m_0 \oplus m_1 \oplus k_1
\oplus k_2 = m_0 \oplus k_1 \oplus k_2$\\
So, the values we have are: $m_0$, $m_1$, $m_2 = m_0 \oplus m_1 \oplus k_1$, and
$m_3 = m_0 \oplus k_1 \oplus k_2$.  From these it is trivial to retrieve $k$.
First, $k_1 = m_2 \oplus m_0 \oplus m_1$, and then $k_2 = m_3 \oplus m_0 \oplus
k_1$.  Putting $k_1$ and $k_2$ together we get $k$.
% 4. (c)
\item Similar to (b), we'll walk through the Feistel ladder to see what the
result of the encryption is.  Again, one round of encryption has the behaviour:
$(m_{i-1}, m_i) \rightarrow (m_i, m_{i+1})$, with $m_{i+1} = m_{i-1} \oplus m_i
\oplus k_i$.
\begin{center}
\begin{tabular}{| c || c | c |}
\hline
\textbf{Round} & \boldmath$m_i$ & \boldmath$m_{i+1}$\\ \hline
Begin & $m_0$ & $m_1$\\ \hline
Round 1 & $m_1$ & $m_0 \oplus m_1 \oplus k_1$\\ \hline
Round 2 & $m_0 \oplus m_1 \oplus k_1$ & $m_0 \oplus k_1 \oplus k_2$\\ \hline
Round 3 & $m_0 \oplus k_1 \oplus k_2$ & $m_1 \oplus k_2 \oplus k_3$\\ \hline
\end{tabular}
\begin{tabular}{c}
\\
\\
\\
$m_1$ cancels out\\
$m_0$ cancels out\\
\end{tabular}
\end{center}

So our givens are $m_0$, $m_1$, $m_0 \oplus k_1 \oplus k_2$, and $m_1 \oplus k_2
\oplus k_3$.  In order to achieve the goal, we only need $k_1 \oplus k_2$ which
is easily obtained from $(m_0 \oplus k_1 \oplus k_2) \oplus m_0$, which were
both given.  Now, assume that $k_1$ is the bitstring $b_1 b_2 \dots b_{128}$,
which means that $k_2$ is $b_{128} b_1 b_2 \dots$.  This means that $k_1 \oplus
k_2$, which we have, is the bitstring $(b_1 \oplus b_{128})(b_1 \oplus b_2)(b_2
\oplus b_3)\dots(b_i \oplus b_{i+1})\dots(b_{127}\oplus b_{128})$.\\
If we knew $b_1 = 0$, then we could find $b_2$, since $b_1 \oplus b_2$ is
the second bit of $k_1 \oplus k_2$.  Similarily, we could then find $b_3$ since
we have $b_2$, and the third bit of $k_1 \oplus k_2$.  We could repeat this
process to retrieve the entirety of $k$.  Therefore, $k$ can be determined to be
one of two values based on whether $b_1 = 0$ or $b_1 = 1$.

\end{list}

\end{list}
\end{document}
