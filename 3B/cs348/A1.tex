\documentclass[12pt]{article}
% Preamble
\usepackage[left=2cm,top=1cm,right=2cm,bottom=1cm,nohead,nofoot]{geometry}
\usepackage{listings}

% Header
\title{CS348 Assignment 1}
\author{Daniel Burstyn (20206120)}
\date{Oct 23, 2009}

% Body
\begin{document}
\maketitle
\section*{Question 1 - SQL Queries}
\subsection*{1.}

\begin{lstlisting}[language=sql]
select distinct p.pnum, p.pname, p.dept \
from professor p, class c, schedule s \
where p.pnum = c.pnum \
  and c.cnum = s.cnum and c.term = s.term and c.section = s.section \
  and s.day = 'Tuesday' \
  and not exists ( select * from mark m where m.cnum = c.cnum \
                                          and m.term = c.term \
                                          and m.section = c.section )
\end{lstlisting}
\vspace{1em}
\subsection*{2.}

\begin{lstlisting}[language=sql]
select count(distinct p.pnum) as NumProfessors \
from professor p, class c \
where p.pnum = c.pnum \
  and c.cnum = 'CS348' \
  and exists ( select * from mark m where m.cnum = c.cnum \
                                      and m.term = c.term \
                                      and m.section = c.section )
\end{lstlisting}
\vspace{1em}
\subsection*{3.}

\begin{lstlisting}[language=sql]
select c.cnum, c.cname, m.grade \
from course c, class cl, enrollment e, mark m \
where c.cnum = cl.cnum \
  and e.cnum = cl.cnum and e.term = cl.term and e.section = cl.section \
  and e.cnum = m.cnum and e.term = m.term \
  and e.section = m.section and e.snum = m.snum \
  and m.snum = 1224
\end{lstlisting}
\vspace{1em}
\pagebreak    % Force page break here for nice splitting
\subsection*{4.}

\begin{lstlisting}[language=sql]
select * from student \
except \
select distinct s.* \
from professor p, class c, enrollment e, mark m, student s \
where p.pnum = c.pnum \
  and p.dept != 'Philosophy' \
  and e.cnum = c.cnum and e.term = c.term and e.section = c.section \
  and e.cnum = m.cnum and e.term = m.term \
  and e.section = m.section and e.snum = m.snum \
  and s.snum = m.snum \
  and m.grade < 92
\end{lstlisting}
\vspace{1em}
\subsection*{5.}

\begin{lstlisting}[language=sql]
select pnum, pname, dept \
from professor p \
except \
select distinct p.pnum, p.pname, p.dept \
from professor p, class c, schedule s \
where p.pnum = c.pnum \
  and c.cnum = s.cnum and c.term = s.term and c.section = s.section \
  and ( s.day = 'Monday' or s.day = 'Friday' ) \
  and not exists ( select * from mark m where m.cnum = c.cnum \
                                          and m.term = c.term \
                                          and m.section = c.section )
\end{lstlisting}
\vspace{1em}
\subsection*{6.}

\begin{lstlisting}[language=sql]
select c.cnum, c.term, c.section, p.pnum, p.pname \
from professor p, class c \
where p.pnum = c.pnum \
  and exists ( select * from mark m where m.cnum = c.cnum \
                                      and m.term = c.term \
                                      and m.section = c.section )
\end{lstlisting}
\vspace{1em}
\pagebreak  % Force page break here
\subsection*{7.}

\begin{lstlisting}[language=sql]
with Enrollments(cnum, enrollment) as ( \
  select e.cnum, count(e.snum) \
  from class cl, enrollment e \
  where e.cnum = cl.cnum and e.term = cl.term and e.section = cl.section \
  group by e.cnum \
  union \
  select c.cnum, 0 \
  from course c \
  where not exists ( select * from class where class.cnum = c.cnum) ) \
select * from Enrollments \
except \
select e1.cnum, e1.enrollment \
from Enrollments e1, Enrollments e2, Enrollments e3, Enrollments e4 \
where e1.enrollment > e2.enrollment and \
      e2.enrollment > e3.enrollment and \
      e3.enrollment > e4.enrollment
\end{lstlisting}
\vspace{1em}
\subsection*{8.}

\begin{lstlisting}[language=sql]
with Enrollments(pnum, pname, cnum, term, section, enrollment) as ( \
  select p.pnum, p.pname, c.cnum, c.term, c.section, count(s.snum) \
  from professor p, class c, enrollment e, student s \
  where p.pnum = c.pnum \
    and e.cnum = c.cnum and e.term = c.term and e.section = c.section \
    and e.snum = s.snum \
    and s.year = 2 \
    and not exists ( select * from mark m where m.cnum = c.cnum \
                                            and m.term = c.term \
                                            and m.section = c.section ) \
    group by p.pnum, p.pname, c.cnum, c.term, c.section) \
(select * from Enrollments \
 union \
 select p.pnum, p.pname, c.cnum, c.term, c.section, 0 as enrollment \
 from professor p, class c \
 where p.pnum = c.pnum \
   and not exists (select * from Enrollments e \
                     where c.cnum = e.cnum \
                       and c.term = e.term \
                       and c.section = e.section) \
   and not exists ( select * from mark m where m.cnum = c.cnum \
                                           and m.term = c.term \
                                           and m.section = c.section ) \
) order by pname, pnum, cnum, term, section
\end{lstlisting}
\vspace{1em}
\pagebreak  % Force pagebreak here
\subsection*{9.}

\begin{lstlisting}[language=sql]
select p.pnum, p.pname, c.cnum, co.cname, c.term, c.section, \
       min(m.grade) as min, max(m.grade) as max \
from professor p, course co, class c, mark m \
where p.pnum = c.pnum \
  and co.cnum = c.cnum \
  and c.cnum = m.cnum and c.term = m.term and c.section = m.section \
  and p.dept = 'Statistics' \
  and exists ( select * from mark m where m.cnum = c.cnum \
                                      and m.term = c.term \
                                      and m.section = c.section ) \
  group by p.pnum, p.pname, c.cnum, co.cname, c.term, c.section
\end{lstlisting}
\vspace{1em}
\subsection*{10.}

\begin{lstlisting}[language=sql]
with Departments(dept) as ( \
  select distinct p.dept \
  from professor p, class c \
  where p.pnum = c.pnum \
    and exists ( select * from mark m where m.cnum = c.cnum \
                                        and m.term = c.term \
                                        and m.section = c.section ) \
  group by p.pnum, p.dept, c.term \
  having count(c.section) > 1) \
select (count(d.dept) * 100.0) / count(distinct p.dept) as Percentage \
from Departments d, professor p
\end{lstlisting}

\pagebreak
\section*{Question 2 - Relational Algebra}

\subsection*{1.}
Construct pastClasses table containing only past classes with the same schema as the class table.\\
$\rho(pastClasses\ (),\ \pi_{cnum, term, section, pnum}(class \bowtie mark))$ \\
\\
$\pi_{pnum, pname, dept}(\sigma_{day=Tuesday}( professor \bowtie (class - pastClasses) \bowtie schedule))$

\subsection*{3.}
$\pi_{cnum, cname, grade}(\sigma_{snum=1224}(student) \bowtie mark \bowtie course)$

\subsection*{4.}
Start by filtering out all professors in philosophy.  Then join with student,
mark, and class to get all students marks from professors not in philosophy.
Finally, pull out any students with any mark lower than 92, and use a set
subtraction to get the desired result. \\
$student - \pi_{snum, sname, year}(\sigma_{grade < 92}(student \bowtie mark \bowtie class \bowtie (\sigma_{dept!=Philosophy}(professor))))$

\subsection*{6.}
$\pi_{cnum, term, section, pnum, pname}(professor \bowtie (class \bowtie mark))$

\subsection*{9.}
statsMarks is a table with the same schema as mark, but only has marks from classes taught by statistics professors.\\
$\rho(statsMarks\ (),\ \pi_{cnum, term, section, grade}(\sigma_{dept=Statistics}(professor) \bowtie class \bowtie mark))$ \\
\\
Find the minimum mark for each section using the marks from statsMarks. \\
$\rho(minimums\ (grade\rightarrow min),\ statsMarks - (statsMarks_1 \bowtie_{grade_1 > grade_2} statsMarks_2))$ \\
\\
Find the maximum mark for each section using the marks from statsMarks. \\
$\rho(maximums\ (grade\rightarrow max),\ statsMarks - (statsMarks_1 \bowtie_{grade_1 < grade_2} statsMarks_2))$ \\
\\
Since we used the rename operator, we can simply join all these tables to get the result. \\
$professor \bowtie class \bowtie course \bowtie maximums \bowtie minimums$

\end{document}
