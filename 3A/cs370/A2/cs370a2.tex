\documentclass[12pt]{article}
% Preamble
\usepackage[left=2cm,top=1cm,right=2cm,nohead,nofoot]{geometry}

% Header
\title{CS370 Assignment 2}
\author{Daniel Burstyn (20206120)}
\date{Feb 16, 2009}

% Body
\begin{document}
\maketitle
\newcounter{count}
\newcounter{subcount}[count]
\section{Analytic Problems}
\begin{list}{\arabic{count}.}{\usecounter{count}}

% Question 1: Suspension
\item 
\begin{list}{\alph{subcount})}{\usecounter{subcount}}

% 1. a)
\item 
$x'_1(t) = w_1(t)$\\
$x'_2(t) = w_2(t)$\\
$w'_1(t) + c_1(w_1(t) - w_2(t)) + k_1(x_1(t) -  x_2(t)) + k_2x_1(t) = sin(t)$\\
$w'_2(t) + c_2(x_1(t) - x_2(t)) - c_1(w_1(t) - w_2(t)) = 0$\\
With initial conditions: $x_1(0)=x_2(0)=w_1(0)=w_2(0)=0$

% 1. b)
\item
We need to rearrange our functions like so:\\
$w'_1(t) = sin(t) - c_1(w_1(t) - w_2(t)) - k_1(x_1(t) -  x_2(t)) - k_2x_1(t)$\\
$w'_2(t) = c_1(w_1(t) - w_2(t)) - c_2(x_1(t) - x_2(t))$\\
Thus, we get our vectors:\\
$y = \left(
\begin{array}{c}
x_1(t)\\ x_2(t)\\ w_1(t)\\ w_2(t)
\end{array} \right),
f = \left(
\begin{array}{l}
w_1(t)\\
w_2(t)\\
sin(t) - c_1(w_1(t) - w_2(t)) - k_1(x_1(t) -  x_2(t)) - k_2x_1(t)\\
c_1(w_1(t) - w_2(t)) - c_2(x_1(t) - x_2(t))\\
\end{array} \right)$
\end{list}

% Question 2.
\item For this question we have $y'(t) = f(t,y) = -\lambda y$ to test the
stability for:\\
$y_{n+1} = y_n + \frac{h}{4}[f(t_n,y_n) + 3f(t_n + \frac{2h}{3},
y_n + \frac{2}{3}hf(t_n,y_n))]$
\begin{eqnarray*}
y_{n+1} &=& y_n + \frac{h}{4}\left[f(t_n,y_n) + 3f\left(t_n + \frac{2h}{3},
y_n + \frac{2}{3}hf(t_n,y_n)\right)\right]\\
&=& y_n + \frac{h}{4}\left[-\lambda y_n -3\lambda\left(y_n -
\frac{2}{3}h\lambda y_n \right) \right]\\
&=& y_n\left[1 - \frac{h\lambda}{4} - \frac{3h\lambda}{4} +
\frac{h^2\lambda^2}{2}\right]\\
&=& y_n \left[1 - h\lambda + \frac{h^2\lambda^2}{2}\right]\\
&=& y_{n-1} \left[1 - h\lambda + \frac{h^2\lambda^2}{2}\right]^2\\
&=& \dots\\
&=& y_0 \left[1 - h\lambda + \frac{h^2\lambda^2}{2}\right]^{n+1}\\
\end{eqnarray*}
\newpage
Thus we can see that the solution decays to zero as $n \rightarrow \infty$ if
and only if $|1 - h\lambda + \frac{h^2\lambda^2}{2}| \le 1$
Solving this inequality we get:
\begin{eqnarray*}
-h\lambda + h^2\frac{\lambda^2}{2} \le 0 &\Rightarrow& 
h \ge 0 \textrm{ and } h \le \frac{2}{\lambda} \\
&\textrm{ and }&\\
0 \le 2-h\lambda + h^2\frac{\lambda^2}{2} &\Rightarrow& 
\textrm{Nothing, since this is always true}\\
\end{eqnarray*}
Therefore, we can conclude that this method is conditionally stable for $h \in
\left[0, \frac{2}{\lambda}\right]$.

% Question 3.
\item 
\begin{list}{\alph{subcount})}{\usecounter{subcount}}

% 3. a)
\item For reference, we have the taylor series, which is:\\
$y(t_{n+1}) = y(t_n) + y'(t_n)h + y''(t_n)\frac{h^2}{2} +
y'''(t_n)\frac{h^3}{6} + \dots$\\
Our function gives us:
\begin{eqnarray*}
\frac{y_{n+1} - y_{n-1}}{2h} &=& f(t_n,y_n)\\
y(t_{n+1}) &=& y(t_{n-1}) + 2hy'(t_n)\\
&=& \left(y(t_n) - y'(t_n)h + y''(t_n)\frac{h^2}{2} - y'''(t_n)\frac{h^3}{6} +
\dots\right)\\
&& + 2h\left(y'(t_n) + y''(t_n)h + y'''(t_n)\frac{h}{2} + \dots\right)\\
&=& y(t_n) + y'(t_n)h + y''(t_n)\frac{5h^2}{2} + \dots\\
\end{eqnarray*}
We can see that this differs from the taylor expansion in the $y''(t_n)$ term,
thus we can conclude that the local error is $O(h^2)$.

% 3. b)
\item Like above, our method is $y_{n+1} = y_{n-1} + 2hy'_n$.  We will use
$y'(t) = f(t,y) = -\lambda y$, so that we have:
\[y_{n+1} = y_{n-1} - 2h\lambda y_n\]
Applying perturbation analysis, we get:
\[e_{n+1} = e_{n-1} - 2h\lambda e_n\]
Solve this recurrence relation by setting $e_n = \mu^n$:
\[\mu^2 + 2h\lambda\mu - 1 = 0\]
which gives
\begin{eqnarray*}
\mu &=& \frac{-2h\lambda \pm \sqrt{4h^2\lambda^2 + 4}}{2}\\
&=& -h\lambda \pm \sqrt{h^2\lambda^2 + 1}
\end{eqnarray*}
After some algebra, we can show that this method is \emph{unconditionally
unstable}.
\end{list}

\newpage
\section{Programming Problems}
% Question 4.
\item We find that given the values set in the assingment sheet, the pursuer
will catch the target at $t = 12.9226$.  See attached plot, and documented code.

% Question 5.
\item The resulting table:
\[
\begin{tabular}{|c|c|c|}
\hline
\textbf{Tolerance} & \textbf{T} & \textbf{No. of Ftn Evals}\\ \hline
$10^{-3}$ & 10.1699 & 326\\ \hline
$10^{-4}$ & 12.9243 & 558\\ \hline
$10^{-5}$ & 12.9225 & 523\\ \hline
$10^{-6}$ & 12.9226 & 703\\ \hline
$10^{-7}$ & 12.9225 & 917\\ \hline
$10^{-8}$ & 12.9225 & 1149\\ \hline
$10^{-9}$ & 12.9225 & 1573\\ \hline
\end{tabular}
\]

We can plainly see that as we lower the error tolerance, the number of function
evaluations increases.  Since our tolerance for error is so low it takes a much
larger number of evaluations to find an acceptable solution.  We can also see
that, with the exception of the first entry, our $T$ values are approximately
equal.  This makes sense, since the solution to the system should not depend on
the error tolerance, for tolerance that is sufficiently small.  Of course, this
shows that $10^{-3}$ is not small enough, since we get a solution that is
obviously incorrect.

% Question 6.
\item Using the same table format as question 5, we get the table:
\[
\begin{tabular}{|c|c|c|}
\hline
\textbf{Tolerance} & \textbf{T} & \textbf{No. of Ftn Evals}\\ \hline
$10^{-5}$ & 24.3609 & 15325\\ \hline
$10^{-6}$ & 18.6414 & 13009\\ \hline
$10^{-7}$ & 12.9225 & 12793\\ \hline
\end{tabular}
\]

Here we immediately notice that the number of function evaluations is much
larger than in 5.  This is because ode45 is a nonstiff solver, and thus uses
many more evaluations to arrive at a solution.  Since we have so many
evaluations, our global errors get much higher, and so we need a lower tolerance
in order to arrive at the correct solution.  This explains the other solutions
times that we got using ode45.

% Question 7.
\item The evasion strategy I chose to implement is as follows:\\
If the distance between the pursuer and the target is less than some constant,
$d_0$, then the target should make a sharp turn in the opposite direction as
normal.  Besides that, the code is the same as the other programming problems.

The reason for this evasion strategy is that, the target can turn more sharply
than the pursuer, so when the pursuer gets close, we make a sharp turn away from
where the pursuer expects us to be.  This cause the pursuer to end up generally
further away, and facing the wrong direction, allowing the target to escape.

Setting $d_0$ is a matter of trial and error, and I've attached the plots for a
few possibilities.

\end{list}
\end{document}
