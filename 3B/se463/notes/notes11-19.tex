\documentclass[12pt]{article}
\usepackage{ulem}
\begin{document}

\section{Temporal Logic}

\subsection{States + THe Universe of Discourse}
States in state machii diagrams (SMDs) are henceforth referred to as "control states"

OUr notion of state innogic includes
- values of all variables
- existence of all "living" objects + ther references/ptrs
- control states in SMD
- event queue(of external events)

\subsection{Logic as a Constraint Laguage}

Predicate logic can be used to describe properties of a system by \sout{constraining} describing values of variables in a particular state: state formulae

In an execution, values of variables change so state formulae can change truthiness.

int main(...)
int $x, y$;
$s_0$    $x=y=0$;
$s_1$    $y=3$;
$s_2$    $x=5$;
\}

\begin{tabular}{c|ccl}
& x & y & ($x \ge y$)(p) <-- state formulae \\
$s_0$ & 0 & 0 & true \\
$s_1$ & 0 & 3 & false \\
$s_2$ & 5 & 3 & true \\
\end{tabular}

We evaluate a stateformula p with respect to an execution $\sigma$ which represents the sequence of program states for that particular execution $\sigma \equiv s_0s_1s_2$

A state s atisfies a predicate p if p is true in state s
Write as

\end{document}
