\documentclass {article}
\usepackage{fullpage}

\title{CS488 Final Project Proposal}
\author{Daniel Burstyn (20206120)}
\date{Nov 20, 2009}

\begin{document}
\maketitle
\newpage

\noindent{\Large \bf Final Project:}
\section{Purpose}
To implement a number of advanced ray tracer features and model an attractive
scene.

\section{Statement}
The scene I plan to render is that of the top of a marble bathroom sink.  On the
sink there will be a clear glass containing a toothbrush, and sink stopper.
The faucet will be running, and the sink will be full of water.  The wall behind
the sink will have a simple tile texture, and a mirror.

The sink stopper is a cut off cone that will require a cone primitive and CSG to
make.  A torus will also be used as the metal loop at the top of the stopper.
The glass and toothbrush will require refraction.

The water in the sink will be rippled and require bump mapping.  The sink and
faucet will be constructed with CSG.  The mirror and sink will both be
reflective, and texture mapping will be used for the tiles, and running water.

\section{Technical Outline}
\subsection{Additional Primitives}
The additional primitives I plan to add are cones, cylinders, and tori.  Cones
and cylinders are conceptually quite similar to spheres as we can calculate the
intersection of the ray using their simple geometric formulae.\\
Cylinders: $x^2 + z^2 = 1$ (Oriented on the $y$ axis)\\
Cones: $x^2 + z^2 = y^2$ (Oriented on the $y$ axis)\\
Cylinders and cones also need an additional check on the values since they are
not supposed to be infinite.  The equation for a torus is much more complex.\\
Tori: $(\sqrt{x^2 + y^2} - R) + z^2 = r^2$ (Oriented on the $z$ axis)\\
Tori are quartic and can have up to 4 intersections with a ray, so we must use a
quartic root finder to find the intersection points.  Additionally, finding the
normal of a torus is quite complicated, but it can be done by finding the
tangent vector along the torus circle, and the tangent of the cross-section
circle, and the cross product of those vectors gives the normal.\\
The lua parser needs a simple extension to be able to construct these new
primitives, but will be very similar to the existing cube and sphere ones.

\subsection{Texture Mapping and Bump Mapping}
We will first need to extend the lua parser to allow materials to have a bump
map and a texture map as part of them.  These are both images, and are used in
different but similar ways.

To do either mapping, we will need a way to take an intersection point on a
primitive and map it to a specific pixel on the map image.  This is simple for
polygons, but much more difficult for primitives like cones and spheres.  Each
primitive will need a function that can do this mapping.  To map a point on a
conic into $x,y$ coordinates for the map, the angles between a set vector, and
the intersection point vector, and some tricky trigonometry and arclength
equations can be used.

Once we have $x$ and $y$ coordinates, we can lookup that coordinate in the maps.
For texture mapping, we simple set the colour of the point to be that of the
colour in the map.  For bump mapping, we use the RGB values of the bump map to
perturb the normal of that point in $x, y$, and $z$.

\subsection{Constructive Solid Geometry}
Once again we must define new functions for our lua scripts so that we can make
boolean operation nodes that are the cornerstone of CSG.  The boolean nodes will
take two nodes as children and perform a boolean operation on them - either
intersection, union, or difference.

In order to do a ray intersection with boolean node, we must do a ray intersect
with both children, but instead of returning a single intersection point we
return an entire line segment (or multiple segments in the case of a torus),
then the boolean node will perform the boolean operation on the segments to
determine the final intersection point of the object.

We also have to be careful because in the case of a difference, we may
have to flip the direction of the normal for shading.

\subsection{Glossy Reflections}
Mirror reflections can be implemented be recursively ray tracing.  When a ray
hits a shiny object, we cast a ray in the mirror direction and add some fraction
of those results to the lighting calculation of the object originally hit.

Glossy reflections are a simple extension to mirror reflections.  Instead of
casting the secondary ray in the mirror direction, we first perturb the ray
slightly at random. The result is a glossy looking reflection.

\subsection{Refraction}
Refraction is implemented somewhat similarily to reflection.  When the ray hits
a refractive object, we cast a new ray in a slightly modified direction.  The
new direction vector can be computed with snell's law: $n_1\sin{\theta_1} =
n_2\sin{\theta_2}$.  This equation is performed for each dimension, relative to
the normal of the object at the intersection point.

Since we have an epsilon to check that our shadow rays don't intersect with the
object they are on, we also need to make sure that this won't affect our
refracted rays.  The refracted ray will intersect with the refractive object
again when it leaves, so we should make sure that this will still work for very
thin clear objects.

\subsection{Hierarchical Bounding Boxes}
Hierarchical bounding boxes are an extension to the bounding boxes used in A4.
Instead of having a non-hierarchical axis-aligned bounding box around
primitives, we can extend this to have bounding boxes placed arbitrarily
throughout the scene graph.  Further, these boxes can be rotated, and have
non-uniform scales applied to them.

Since we can place these bounding boxes arbitrarily, we can have them contain a
number of primitives and not just a single one.  We can also expand them to
produce bounding boxes for objects made through CSG.  These bounding boxes
reduce the number of intersections that need to be done in the usual case that a
ray wouldn't hit any part of the object.

Since we don't want the user to have to specify all the bounding boxes, we will
instead place bounding box nodes in the scene hierarchy by doing a pass of the
tree before we start rendering.  It would seem logical to place bounding boxes
above SceneNodes, Polygon Meshes, and CSG objects.  It is obvious that bounding
boxes for meshes and CSG will reduce the number of intersections that need to be
done.  Placing bounding boxes around SceneNodes is done since it is common
practice to construct complex objects out of primitives, and put them together
using a SceneNode so they can be transformed all at once, and instantiated
multiple times.  It is logical that this grouping of objects would likely
benefit from a bounding box.

\subsection{Photon Mapping with kD-Trees}
Photon mapping is done by, before rendering, casting many rays---photons---away
from the lights in the scene.  The photons bounce off reflective surfaces until
they reach a diffuse surface where they are stored in a photon map.  The photons
are also obviously affected by refraction as explained above.  Photon
mapping, although computationally intensive, can produce effects that regular
raytracing cannot.

As mentioned above, when the photon hits a diffuse surface, it must be stored in
a photon map.  This photon map is a spatial data structure in nature and for the
purposes of ray tracing, we want fast insert and fast find-density.  To do this
we will implement a kD-tree which can efficiently find the density of photons in
a close area.

A kD-tree is a k-dimensional tree that subdivides space using hyperplanes.  At
each level of the tree, space is divided by a hyperplane along an axis, where
the axis of division rotates at each level of the tree.  To find which plane is
used to divide the space, a nieve method can be used to determine the middle of
the space with respect to all of the objects in that space already.

\subsection{Caustics}
Now that we have constructed a photon map, we can use it do get some neat
lighting effects.  For this project, I plan to do caustics.  Caustics are the
bright spots of light that appear when light is shone through a curved
transparent object like a cylinder or a sphere.  This is one of the effects that
can not easily be achieved through normal ray tracing.

In order to get caustics, we will modify our lighting calculations by adding in
an irradience estimate based on the density of photons in the photon map at the
point of intersection for the ray.  Since the photon map is stored as a kD-tree,
finding the density of photons at any given point is a relatively cheap
operation.

\section{Bibliography}
        1. Most information will come directly from the relevant chapters of the
        provided course notes.\\
        2. http://en.wikipedia.org/wiki/Procedural\_texture\\
        3. people.scs.carleton.ca/~mould/courses/3501/procedural.pptx\\
        4. www.cs.ucdavis.edu/~amenta/s06/findnorm.pdf\\

\newpage


\noindent{\Large\bf Objectives:}

{\hfill{\bf Full UserID:\rule{2in}{.1mm}}\hfill{\bf Student ID:\rule{2in}{.1mm}}\hfill}

\begin{description}
     \item[\_\_\_]  Additional primitives cylinder, cone, and torus are added.

     \item[\_\_\_]  Texture mapping is added.

     \item[\_\_\_]  Bump mapping is added.

     \item[\_\_\_]  Constructive solid geometry for primitives is added.

     \item[\_\_\_]  Glossy Reflections are added.

     \item[\_\_\_]  Refraction is added.

     \item[\_\_\_]  Hierarchical bounding boxes for improved efficiency.

     \item[\_\_\_]  Photon mapping with kD-trees.

     \item[\_\_\_]  Caustics using photon mapping.

     \item[\_\_\_]  A final scene is modelled as described above.
\end{description}
\end{document}
