\documentclass {article}
\usepackage{fullpage}

\title{CS488 Initial Project Proposal}
\author{Daniel Burstyn (20206120)}
\date{Nov 10, 2009}

\begin{document}
\maketitle
\newpage

\noindent{\Large \bf Final Project:}
\begin{description}
\item[Purpose]:\\
	To implement a number of advanced ray tracer features and model an
        attractive scene.

\item[Statement]:\\
        The scene I plan to render is that of the top of a marble bathroom sink.
        On the sink there will be a clear glass containing a thoothbrush, and
        sink stopper.  The faucet will be running, and the sink will be full of
        water.  The wall behind the sink will have a simple tile texture, and a
        mirror.

        The sink stopper is a cut off cone that will require a cone primitive
        and CSG to make.  A torus will also be used as the metal loop at the top
        of the stopper.  The glass and toothbrush will require refraction.

        The water in the sink will be rippled and require bump mapping.  The
        sink and faucet will be constructed with CSG.  The mirror and sink will
        both be reflective, and texture mapping will be used for the tiles, and
        running water.

\item[Technical Outline]:\\
        The additional primitives are very straight forward and will require
        additional intersection computations.

        Texture mapping will be done simply by mapping the pixel on the screen
        to a pixel on the polygon, and then onto the texture map to determine
        the right colour of the pixel.

        Bump mapping is done by perturbing the normals of a polygon according to
        a map.  These changed normals will affect the lighting calculations and
        cause the desired effect.

        For CSG, we must store both member primitives under a boolean operation
        in our hierarchy.  Intersection for the combined object can be done by
        doing ray intersect for both primitives, and performing the appropriate
        boolean operations on the resulting line segments.

        Glossy reflections are a simple extension to mirror reflections.
        Instead of casting a secondary ray in the mirror direction, we first
        perturb the ray slightly at random to get a glossy effect.

        Refraction can be done by changing the direction of the cast ray through
        a clear object.  The direction change can be computed using snell's law.

        To get soft shadows, when we cast our shadow rays to the lights we can
        cast many rays to random points on an area light.  Averaging these
        results will produce soft shadows.

        To get a marble texture for the sink, we can use a procedural generation
        of the 3D texture.  Functions exist to generate marble texture with
        perlin noise.

        Bounding boxes can be used to decrease the number of intersection
        calculations that need to be performed.  When we cast a ray, we check
        whether it intersects any bounding boxes intead of every polygon in the
        scene.  This is much cheaper improves render speed.

\item[Bibliography]:\\
        1. Most information will come directly from the relevant chapters of the
        provided course notes.\\
        2. http://en.wikipedia.org/wiki/Procedural\_texture\\
        3. people.scs.carleton.ca/~mould/courses/3501/procedural.pptx\\

\end{description}
\newpage


\noindent{\Large\bf Objectives:}

{\hfill{\bf Full UserID:\rule{2in}{.1mm}}\hfill{\bf Student ID:\rule{2in}{.1mm}}\hfill}

\begin{description}
     \item[\_\_\_]  Additional primitives cylinder, cone, and torus are added.

     \item[\_\_\_]  Texture mapping is added.

     \item[\_\_\_]  Bump mapping is added.

     \item[\_\_\_]  Constructive solid geometry for primitives is added.

     \item[\_\_\_]  Glossy Reflections are added.

     \item[\_\_\_]  Refraction is added

     \item[\_\_\_]  Soft shadows from area light sources are added.

     \item[\_\_\_]  Procedural solid texture mapping for marble.

     \item[\_\_\_]  Bounding boxes are used for improved efficiency.

     \item[\_\_\_]  A final scene is modelled as described above.
\end{description}
\end{document}
