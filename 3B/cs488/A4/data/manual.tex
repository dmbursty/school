\documentclass[12pt]{article}
% Preamble
\usepackage[left=2cm,top=2cm,right=2cm,nohead,nofoot]{geometry}
\pagestyle{empty}
 
% Header
\title{CS488 Assignment 4}
\author{Daniel Burstyn (dmbursty) (20206120)}
\date{Nov 16, 2009}
 
% Body
\begin{document}
\begin{titlepage}
\maketitle
\thispagestyle{empty}
\end{titlepage}
\newpage

\begin{center}
\begin{Huge}
Manual
\end{Huge}
\end{center}

\section{Extra Feature - Antialiasing}
For the ray tracer assignment, I chose antialiasing to be my extra objective.
To achieve the antialiasing, I cast 15 rays to random points within each pixel
and then average the colour results of the 15 points.  The default number of
rays cast is 15, but that value can be changed at compile time.  The first ray
cast will always be to the middle of the pixel, so setting the ANTIALIAS
variable to 1 will act as normal.

\section{Bounding Volumes}
For this assignment I implemented only bounding (hierarchical) boxes for polygon
meshes.  When the call to render() is made, the ray tracer will do one pass
through the scene tree and make bounding boxes where necessary.  When a geometry
node finds it has a primitive that supports a bounding volume, it will construct
a bounding node that has a pointer to the geometry node, and a new geometry node
containing the bounding volume.  The parent of the geometry node will then swap
it's children to maintain the structure of the tree, with the bounding node
above the geometry node.

I chose this approach because it his highly extensible, as bounding nodes can be
placed anywhere in the scene graph (useful for CSG for the project) and the
bounding volume can be any geometry node itself.

\section{Unique Scene}
The unique scene in sample.png has a number of items of interest.

The barn is modeled with 5 boxes for the base, two polygon meshes for the roof,
and three separate instances of the same mesh for the window and door.  The barn
is hierarchical and can be easily transformed.

The UFO is made up of two spheres and is also hierarchical.  The UFO is made to
be shiny to show off phong lighting.  A light is placed at the bottom of the UFO
to demonstrate coloured lights and interesting shadow effects.

The cow model was provided, and three of these cows are placed for humourous
effect.

\end{document}
